\documentclass{beamer}
\usepackage{listings}
% to choose the documentclass and theme
% https://mpetroff.net/files/beamer-theme-matrix/
\usetheme{Szeged} 
\usecolortheme{beaver}
\usepackage{color}
% % block styles
\setbeamertemplate{blocks}[rounded][shadow=true]
\title{Exploring the variability of snow in Iceland}
 \subtitle{focusing on Fagradalsfjall area }
 \author{Carlotta Zanetti}
 \date{\small{ Link to my GitHub account:\\ \url{https://github.com/carlottazanetti}}}
 \institute{Alma Mater Studiorum UniBO}

 

 \begin{document}
\maketitle

\AtBeginSection[] 
{
\begin{frame}
\frametitle{Outline}
\tableofcontents[currentsection]
\end{frame}
}


\section{Dataset}

\begin{frame}{Sentinel 2 dataset}
    \small{Why Sentinel 2 dataset?}
     \begin{itemize}
  \item Easy to download from the Copernicus website \url{https://dataspace.copernicus.eu/browser/}
  \item Good resolution 
  %10-60 m
  \item Possibility to look at different channels directly from the website
 \end{itemize}
 \includegraphics[width=0.3\textwidth]{2022-04-19-00_00_2022-04-19-23_59_Sentinel-2_L2A_Highlight_Optimized_Natural_Color_.jpg}
 \includegraphics[width=0.3\textwidth]{2022-04-19-00_00_2022-04-19-23_59_Sentinel-2_L2A_Custom_script (1).jpg}
 \includegraphics[width=0.3\textwidth]{2022-04-19-00_00_2022-04-19-23_59_Sentinel-2_L2A_Custom_script.jpg}
\end{frame}


\section{Variability of snow throughout the seasons}

\begin{frame}{Importing images}
    \small{For each month an image was picked to be representative of the 
    snow cover.}
    \smallskip
        \begin{columns}
        \column{0.5\textwidth}
        \small{The images come from different years since the presence of clouds didn't allow to collect all the data from the same year. \\Also no data was available for December.}
        \column{0.5\textwidth}
        \centering
        \includegraphics[width=.99\textwidth]{2023-11-10-00_00_2023-11-10-23_59_Sentinel-2_L2A_Highlight_Optimized_Natural_Color_.jpg}
    \end{columns}
\end{frame}

\begin{frame}
\includegraphics[width=0.24\textwidth]{jan.jpg}
 \includegraphics[width=0.24\textwidth]{feb.jpg}
 \includegraphics[width=0.24\textwidth]{march.jpg}
 \includegraphics[width=0.24\textwidth]{apr.jpg}
 \includegraphics[width=0.24\textwidth]{may.jpg}
 \includegraphics[width=0.24\textwidth]{june.jpg}
 \includegraphics[width=0.24\textwidth]{july.jpg}
 \includegraphics[width=0.24\textwidth]{aug.jpg}
 \includegraphics[width=0.24\textwidth]{sept.jpg}
 \includegraphics[width=0.24\textwidth]{oct.jpg}
 \includegraphics[width=0.24\textwidth]{nov.jpg} \\
 \tiny{In this case I used an RGB with bands 0.4, 1.6, 2.1 microns to differentiate snow (red) from water clouds (white)}
\end{frame}

\begin{frame}{Snow detection}
    \small{After importing the images with the function rast(), im.classify() from package imageRy was used to classify the images in four levels of energy. }
    \lstinputlisting[basicstyle=\ttfamily\scriptsize,language=R, frame=single, stringstyle=\color{blue}, backgroundcolor=\color{lightgray}]{classify.r} 
    \includegraphics[width=0.9\textwidth]{snow_clusters.jpeg} 
\end{frame}

\begin{frame}{Snow cover}
    \small{For each image with some snow cover, the level corresponding to snow was detected and the percentage of pixels were calculated}
     \lstinputlisting[basicstyle=\ttfamily\scriptsize, language=R, frame=single, stringstyle=\color{blue}, backgroundcolor=\color{lightgray}]{precentage_snow.r} 
\end{frame}

\begin{frame}{Snow cover}
    \small{The results were plotted in a histogram}
    \lstinputlisting[basicstyle=\ttfamily\scriptsize,language=R, frame=single, stringstyle=\color{blue}, backgroundcolor=\color{lightgray}]{histogram.r} 
   \centering
   \includegraphics[width=0.8\textwidth]{snow coverage seasonal cycle.jpeg}
\end{frame}


\section{Variability of snow throughout the years}
\begin{frame}{Importing images}
    \small{In this case, pictures from April 2020, 2022, 2023 were taken.}
    \lstinputlisting[basicstyle=\ttfamily\scriptsize,language=R, commentstyle=\color{darkgray}, frame=single, stringstyle=\color{blue}, backgroundcolor=\color{lightgray}]{years_variability.r} 
    \includegraphics[width=0.99\textwidth]{snow_cover2.jpeg}
\end{frame}

\begin{frame}{Correlation}
    \lstinputlisting[basicstyle=\ttfamily\tiny,language=R, commentstyle=\color{darkgray}, frame=single, stringstyle=\color{blue}, backgroundcolor=\color{lightgray}]{pairs.r} 
    \tiny{Using the function pairs() to investigate the correlation between different years. Let's analyse the reflectances:}
    \begin{itemize}
     \item Pixels with low reflectance represent the ocean
     \item Pixels with high reflectance represent snow
   \end{itemize}   
   \includegraphics[width=0.99\textwidth]{correlations.jpeg}
\end{frame}

\begin{frame}{Snow differences}
    \begin{columns}
        \column{0.35\textwidth}
        \centering
        \small{To highlight the variability between the year 2020 and 2023, their difference was plotted}
        \lstinputlisting[basicstyle=\ttfamily\tiny,language=R, commentstyle=\color{darkgray}, frame=single, stringstyle=\color{blue}, backgroundcolor=\color{lightgray}]{diff.r} 
        \column{0.65\textwidth}
        \centering
        \includegraphics[width=.99\textwidth]{snow_diff.jpeg}
    \end{columns}
    \begin{columns}
        \column{0.4\textwidth}
        \centering
        \small{An RGB with 2020 in the red channel, 2022 in the green channel and 2023 in the blue channel shows where the contribution of each year is stronger}
        \lstinputlisting[basicstyle=\ttfamily\tiny,language=R, commentstyle=\color{darkgray}, frame=single, stringstyle=\color{blue}, backgroundcolor=\color{lightgray}]{rgbdiff.r} 
        \column{0.65\textwidth}
        \centering
        \includegraphics[width=.99\textwidth]{snow_coverRGB.jpeg}
    \end{columns}
\end{frame}


\section{A remarkable phenomenon: Fagradalsfjall volcanic eruption}

\begin{frame}{False color RGB}
    \small{To analyse the Fagradalsfjall volcanic eruption occurred in 2010, I decided to create an RGB using bands 2.1, 0.8 and 0.6 microns. This false color RGB is used to spot volcanic eruptions and fires:}
     \begin{columns}
        \column{0.45\textwidth}
        \centering
        \begin{figure}[h]
            \centering
            \includegraphics[width=.99\textwidth]{screenshotfire.png}
            \caption{example of a fire over Georgia spotted with Modis}
        \end{figure}
        \column{0.65\textwidth}
        \begin{itemize}
  \item Lava (or fire) will appear red due to the strong reflectivity in the 2.1 band
  \item Vegetation will appear green due to the strong reflectivity in the 0.8 channel, hence burnt areas will appear brown 
  \item The smoke plume wil appear cyan, due to the high reflectivity in the visible band
 \end{itemize}    
    \end{columns}
\end{frame}

\begin{frame}{Importing images}
  \includegraphics[width=.99\textwidth]{Rplot.jpeg}
\end{frame}

\begin{frame}{Creating the RGB}
   \lstinputlisting[basicstyle=\ttfamily\tiny,language=R, commentstyle=\color{darkgray}, frame=single, stringstyle=\color{blue}, backgroundcolor=\color{lightgray}]{falsergb.r} 
  \centering
  \includegraphics[width=.8\textwidth]{iceland_eruption_falseRGB.jpeg}
\end{frame}

\begin{frame}{Standard deviation}
  \small{How to decide which band to pick to calculate the standard deviation?\\
  We can perform the Principal Component Analysis to retrieve the pc1, which will be the layer that explains variability the most.}
  \lstinputlisting[basicstyle=\ttfamily\tiny,language=R, commentstyle=\color{darkgray}, frame=single, stringstyle=\color{blue}, backgroundcolor=\color{lightgray}]{pca.r}  
  \includegraphics[width=.8\textwidth]{Screenshot 2024-01-15 123053.png}\\
  \small{By looking at the table, it's easy to recognise which band contributes the most to pc1: it's the 0.8 band.}
\end{frame}

\begin{frame}{Standard deviation}
  \tiny{The standard deviation is calculated using the function focal() combined with the std function. In this case the std is calculated on a moving window of 3x3 pixels}
  \lstinputlisting[basicstyle=\ttfamily\tiny,language=R, commentstyle=\color{darkgray}, frame=single, stringstyle=\color{blue}, backgroundcolor=\color{lightgray}]{std.r}   
  \tiny{The std can also be calculated on the 0.8 band}
  \begin{figure}
    \centering
    \includegraphics[width=.8\textwidth]{std.jpeg}
  \end{figure}
  \tiny{As expected, the two images are similar, since band 0.8 contributes the most to pc1. Values of high std can generally be found on boundaries, indicating high geological variability. }
\end{frame}

\begin{frame}
\begin{center}
{\fontsize{30}{40}\selectfont Thank you!}
\end{center}
\end{frame}
\end{document}
